\section{บทนำ ที่มาของปัญหา}

ในปัจจุบัน แนวคิดของระบบเครือข่ายเหนือพื้นดิน(Non Terrestrial Network - NTN) ถูกพัฒนาและนำเสนออกมาอย่างหลากหลาย
และถูกนำไปปรับใช้กับอุปกรณ์ที่เกี่ยวกับการกระจายสัญญาณเครือข่ายเช่นดาวเทียม โดยแบ่งตามระดับวงโคจรได้แก่ ระดับต่ำ(Low-Earth-Orbit - LEO)/
ระดับกลาง(Medium-Earth-Orbit - MEO) ระดับสูงที่สุด(Geostationary-Earth-Orbit - GEO)
โดยภายในระบบ NTN มีองค์ประกอบสำคัญคือเทคโนโลยีเครือข่ายยานพาหนะไร้คนขับ (High Altitude Platform Station - HAPS)
เนื่องจากมีข้อดีคือปรับใช้ง่าย การตอบสนองรวดเร็วและครอบคลุมในวงกว้าง เพื่อควบคุมและรักษาศักยภาพของเครือข่ายเหล่านี้ 
การออกแบบระบบเครือข่ายรูปแบบใหม่ที่เกิดจากการรวม NTN และระบบเครือข่ายภาคพื้นดิน(Terrestrial Network - TN) 
จึงเป็นสิ่งที่ Developer และ Network Engineer ให้ความสนใจ

หลายงานวิจัยแสดงให้เห็นถึงความท้าทายในการพัฒแนาเทคโนโลยี HAPS เช่นการปรับโครงสร้างตัวยานพาหนะ ปรับปรุงระบบกักเก็บและส่งพลังงาน
รวมไปถึงระบบระบายความร้อน อีกทั้งยังมีการร่วมในภาคธุรกิจในหลายองค์กร เพื่อเร่งการพัฒนา HAPS จำเป็นต้องได้รับความร่วมมือจากหลายองค์กร
จากงานวิจัย \cite[Towers]{High Altitude Platform Systems: Towers in the Skies} 
ได้นำเสนอแนวทางการนำเทคโนโลยี HAPS มาปรับใช้ในชีวิตประจำวัน แบ่งออกได้เป็นหลายสถานการณ์ได้แก่
\begin{itemize}
    \item เครือข่ายแบบเฉพาะ โดยผู้ให้บริการเครือข่ายจะสามารถเข้าถึงตัวอุปกรณ์กระจายสัญญาณ(Platform)ได้แต่เพียงผู้เดียว 
    และจะใช้ Platform ดังกล่าวเพื่อให้บริการสัญญาณเครือข่ายแบบไร้สายแก่ลูกค้า
    \item ใช้งานเครือข่ายร่วมกัน โดย HAPS อาจถูกนำมาใช้เป็นการร่วมทุนของผู้ให้บริการโทรศัพท์มือถือที่เข้าร่วม
    \item หน่วยงานเอกชนรับหน้าที่เป็น Host ทำให้ HAPS เป็น Service Platform สาธารณะประเภทหนึ่ง
    ทุกคนสามารถเข้าถึงได้และมีการเรียกเก็บค่าธรรมเนียมตามปกติ
    \item หน่วยงานรัฐเป็นผู้ควบคุม
    \item แบบผสม(Hybrid) โดยผู้ให้บริการมือถือเป็น Host และมอบสิทธิการเข้าถึง Platform กับผู้ให้บริการรายอื่นแทน
\end{itemize}
แต่ระหว่างขั้นตอนในการพัฒนาย่อมมีปัญหาหลายประเภทเกิดขึ้นเช่นกัน เช่น ปัญหาในการจัดสรรทรัพยากร 
สัญญาณรบกวนที่เกิดจากการรวมสองระบบที่แตกต่างกัน ปัญหาในการจัดสรรทรัพยากรมีสาเหตุมาจากข้อผิดพลาดระหว่างการกระจายสัญญาณรับส่งข้อมูลแบบไดนามิก 
กล่าวคือ ในความเป็นจริงแล้วปริมาณการรับส่งข้อมูลนั้นสามารถเปลี่ยนแปลงได้ตลอดเวลา โดยมีสาเหตุมาจากเทคโนโลยี HAPS
ีเนื่องจากระบบบางส่วนยังไม่สามารถควบคุมให้เสถียรได้ สำหรับเรื่องสัญญาณรบกวนเป็นอีกหนึ่งในปัญหาสำคัญของการเพิ่มประสิทธิภาพของ HAPS เนื่องจากสัญญาณเครือข่ายที่ HAPS เชื่อมต่อกับ Platform
บนโลกนั้นเป็นเพียงสัญญาณไป-กลับเท่านั้น สัญญาณรบกวนจึงเป็นสาเหตุให้เกิดความล่าช้าของเครือข่ายที่ HAPS ควบคุมอยู่

ซึ่งในรายงานฉบับนี้ได้นำเสนอถึงการแก้ไขปัญหาเหล่านี้ด้วยวิธีการแทรกคลื่นสัญญาณอีกประเภทหนึ่ง
(Interference Coordination Method) เข้าไประหว่างคลื่นสัญญาณของ HAPS กับ สถานีเครือข่ายภาคพื้นดิน
และต้องคำนึงถึงการกระจายปริมาณรับส่งข้อมูลเป็นหลัก เนื่องจากในแต่ละพื้นที่มีจำนวนสัญญาณเครือข่ายประเภทไม่เท่ากัน โดยเฉพาะพื้นที่ที่มีความถี่ของสัญญาณทับซ้อน
ผู้ใช้งานเครือข่ายอาจได้รับผลกระทบจากการรบกวนภายในระบบ จึงมีความจำเป็นต้องตรวจสอบขณะที่ดำเนินการแทรกคลื่นสัญญาณเข้าไปอย่างระมัดระวังเพื่อแก้ไขปัญหาเหล่านี้
จากผลการทดสอบแสดงให้เห็นว่าวิธีแทรกคลื่นสัญญาณนั้นมีประสิทธิภาพเหนือกว่าวิธีการจัดสรรทรัพยากรแบบเดิม โดยสังเกตที่ความเปลี่ยนแปลงของปริมาณทรัพยากร