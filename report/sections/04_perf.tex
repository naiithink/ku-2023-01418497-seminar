\section{ผลลัพธ์ประสิทธิภาพการทำงานระหว่าง HAPS และ Terrestrial Networks}
\subsection{วิเคราะห์การนำวิธี ICM เข้าไปใช้ในเทคโนโลยี HAPS}
สาเหตุของสัญญาณรบกวนนั้น เกิดจากข้อจำกัดปริมาณการส่งข้อมูลใน 1 ชุดคำสั่ง(Protocol)
HAPS กระจายสัญญาณครอบคลุมในพื้นที่วงกว้าง จึงทำให้ไม่สามารถแยกแยะความถี่ในแต่ละคลื่นสัญญาณได้
การสื่อสารระหว่างสัญญาณจาก HAPS ไปยังสถานีภาคพื้นดินหลัก(Base Station - BS)
และอุปกรณ์รับสัญญาณของผู้ใช้งานทั่วไป เช่น Smartphone(User Equipment - UE) \ref{fig:04-haps-with-tn} 

\begin{figure}[h]
\centering
\caption[Interference Type]{Analysis for Inter-System Interference} 
\label{fig:04-haps-with-tn} 
\includegraphics[width=0.3\textwidth]{04_haps-with-tn.png}
\end{figure}

\begin{itemize}
    \item Type 1 - เกิดสัญญาณรบกวนเป็นหลักที่ฝั่ง UE
    \item Type 2 - เกิดสัญญาณรบกวนเล็กน้อยที่ฝั่ง H-UE
    \item Type 3 - สัญญาณ Uplink เกิดสัญญาณรบกวนเล็กน้อยที่ฝั่ง H-UE แสดงให้เห็นว่าการปรับทิศทางส่งสัญญาณส่งผล
    ต่อการลดสัญญาณรบกวน
    \item Type 4 - การส่งสัญญาณจากสถานีหลัก(Base Station - BS) ไป HAPS Uplink
\end{itemize}

ผลการทดลองทั้ง 4 Type สรุปได้ว่าสัญญาณที่ส่งจากสถานีหลักไปยัง HAPS ด้วยการปรับมุมและทิศทางสามารถช่วยลดโอกาสที่จะเกิดสัญญาณรบกวนและลดปริมาณทรัพยากรที่ใช้ และประเมินผลออกมาเป็นกราฟดังนี้
\ref{fig:04-haps-compare-downlink}

\begin{figure}[h]
\centering
\caption[HAPS compare with downlink]{Downlink SE performance in terrestrial network (TN) with or without
the interference from HAPS downlink.} 
\label{fig:04-haps-compare-downlink}
\includegraphics[width=0.3\textwidth]{04_haps-compare-downlink.png}
\end{figure}

กราฟแสดงถึงผลที่เกิดจากสัญญาณรบกวนจาก HAPS ไปยังสถานีภาคพื้นดิน(TN) และอุปกรณ์ของผู้ใช้(UE)
ในความถี่และรัศมีครอบคลุมที่แตกต่างกัน(2 GHz, 39 GHz, 0 km., 100 km.) โดยมีตัวชี้วัดคือประสิทธิภาพของสเปกตรัม(Spectral Efficiency - SE)
วัดปริมาณข้อมูลที่ส่งผ่านแบนด์วิธ(bandwidth)ภายในระบบ

\subsection{การปรับใช้}
แบ่งออกได้เป็นหลายกรณี ได้แก่

\begin{description}
    \item Case a - ทดลองในพื้นที่ที่ไม่มีสัญญาณซ้อน(Non-overlapping areas) ผลที่ได้คือจะไม่เกิดสัญญาณรบกวนภายในระบบ
            ทำให้วิธีการเชื่อมต่อแบบเดิมโดย HAPS เป็นศูนย์กลาง(HAPS Stand-Alone) ยังสามารถใช้งานได้ตามปกติ
    \item Case b1 - ทดลองในพื้นที่ที่มีสัญญาณทับซ่อนแต่ไม่มีการรับส่งข้อมูลเครือข่ายภาคพื้นดิน(Overlapping area, No traffic in terrestrial systems)
            ผลที่ได้คือสามารถเปลี่ยนเป็นโหมด Idle เพื่อลดการใช้พลังงานและป้องกันสัญญาณรบกวนที่เข้ามาในระบบ เนื่องจากระยะระหว่าง HAPS กับ User Equipment ห่างกันมาก
    \item Case b2 - ทดลองในพื้นที่ที่มีสัญญาณทับซ้อนแต่ไม่มีการรับส่งข้อมูลทางฝั่ง HAPS(Overlapping areas, No traffic in HAPS Coverage)
            เมื่อปิด HAPS beam และสัญญาณจากอุปกรณ์ของผู้ใช้(Customer premises equipment - CPE) ส่งผลในสัญญาณรบกวนระหว่าง HAPS กับ TN ลดลงอย่างมาก
    \item Case b3 - ทดลองในพื้นที่ที่มีสัญญาณรบกวนและมีการรับส่งข้อมูลทั้งสองเครือข่าย(Both overlapping areas and traffic in HAPS with terrestrial systems)
            ผู้ใช้งานจะได้รับผลกระทบจากทั้งสองระบบ นอกจากนี้ใน Case b3 ยังจำแนกออกเป็น 4 Pattern หลักได้ดังต่อไปนี้ 
            Pattern 1 -> HAPS และ TN กระจายสัญญาณในพื้นที่ที่ครอบคลุมโดยใช้ปริมาณทรัพยากรและสัญญาณความถี่ในช่วงเวลาเดียวกัน หมายความว่าไม่มีการประสานงานระหว่างทั้งสองระบบ 
                อาจส่งผลให้เกิดสัญญาณรบกวนจนส่งผลให้ประสิทธิภาพและการใช้ทรัพยากรลดลง
            Pattern 2 -> HAPS และ TN กระจายสัญญาณเหมือนเดิม แต่ใช้วิธี Inter-Cell Interference Coordination(ICIC) เป็นอีกความเป็นไปได้ในการลดสัญญาณรบกวน
            Pattern 3 -> HAPS cells เปลี่ยนสถานะเป็น Idle และให้ Base Station เป็นตัวกลางหลักแทน โดย Pattern นี้ส่งผลแค่ในกรณีที่รับส่งข้อมูลใน cell ต่ำเท่านั้น
            Pattern 4 -> Base Station เปลี่ยนสถานะเป็น Idle และให้ HAPS จัดการในพื้นที่ที่มีสัญญาณซ้อน โดย Pattern นี้ส่งผลแค่ในกรณีที่ traffic load ใน TN น้อย
\end{description}

โดยรวมแล้วเราสรุปผลในทาง techical terms ออกมาได้ดังตารางต่อไปนี้ \ref{fig:04-haps-evaluation}

\begin{figure}[h]
\centering
\caption[]{Evaluation Assumptions}
\label{fig:04-haps-evaluation} 
\includegraphics[width=0.5\textwidth]{04_haps_evaluation.png}
\end{figure}

\subsection{การประยุกต์ใช้ ICM}
จากงานวิจัยของ \cite[Interference Coordination Method for Integrated HAPS-Terrestrial Networks]{liu2021interference}
นำเสนอถึงวิธีการแทรกคลื่นสัญญาณ(Interference Coordination Method - ICM) เข้าไประหว่างคลื่นสัญญาณของ HAPS กับ สถานีเครือข่ายภาคพื้นดิน โดยมีเงื่อนไขต่อไปนี้
\begin{itemize}
    \item คำนึงถึงการกระจายปริมาณรับส่งข้อมูลเป็นหลัก 
    \item ประสิทธิภาพที่ได้โดยเทียบจากปริมาณพลังงาน,ทรัพยากรที่เสียไปและปริมาณข้อมูลที่เคลื่อนที่ผ่านเครือข่าย ณ เวลาที่กำหนด (Traffic Load)
\end{itemize}
